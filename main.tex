\documentclass[lang=cn,11pt,chinese]{elegantbook}

\title{洛理指南:We Are Luoli Ren}
\subtitle{LuoYang Institute Of Science And Technology The LuoLi Guide}

\institute{Ajia Network Technology Studio}
\date{April 12, 2021}
\author{AjiaErin}
\version{1.0}

\extrainfo{Your college can be wonderful. --- AjiaErin}

\logo{logo.png}
\cover{bg1.jpg}

% 本文档命令
\usepackage{array}
\newcommand{\ccr}[1]{\makecell{{\color{#1}\rule{1cm}{1cm}}}}
% 修改目录深度
\setcounter{tocdepth}{2}

\begin{document}

\maketitle
\frontmatter

\chapter*{特别声明}
\markboth{Introduction}{前言}

这本书是【洛理指南】的伴生产物,\href{https://wiki.iluoli.ren/feeling/start.html}{这里}有一份项目的诞生与初衷,书中所有内容都可以在我的 \href{https://iluoli.ren/}{网站}中找到相对应的文章,PDF文档的优势是在线文档无法比拟的(至少我是这样认为的,不接受反驳),两者我都会力所能及的更新下去,至于能更多久、质量如何,影响因素太多太多。一直以来我不断的在思考一个问题:对于洛理指南来说什么是一个好的文档呢?期间每想到一个点,我都会去折腾一下,以至于忽略了最重要的核心,我的目光也局限于wiki相关的解决方案,忽略了项目启动的初衷,明白了这点思路也就开阔了。

今后,洛理指南只做结构化、碎片化、体系化的内容产出,唯独我缺乏艺术细胞,只愿用拙劣的文笔,撰写内心最真诚的想法和建议,但是个人精力始终有限,未来我会邀请本校的小伙伴一起参与文档编写的工作!在附录我将展示编写人员的名单,如果你也是喜欢折腾,有意愿参与文档编写的工作,\underline{热烈欢迎加入\href{http://mail.qq.com/cgi-bin/qm_share?t=qm_mailme&email=_J_Zl5mSkZm4iYnWm5eV}{我们}!}



\vskip 1.5cm

\begin{flushright}
Ethan Deng\\
February 10, 2020
\end{flushright}

\tableofcontents
%\listofchanges

\mainmatter




\end{document}
